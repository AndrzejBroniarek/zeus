% TEMPLATE for Usenix papers, specifically to meet requirements of
%  USENIX '05
% originally a template for producing IEEE-format articles using LaTeX.
%   written by Matthew Ward, CS Department, Worcester Polytechnic Institute.
% adapted by David Beazley for his excellent SWIG paper in Proceedings,
%   Tcl 96
% turned into a smartass generic template by De Clarke, with thanks to
%   both the above pioneers
% use at your own risk.  Complaints to /dev/null.
% make it two column with no page numbering, default is 10 point

% Munged by Fred Douglis <douglis@research.att.com> 10/97 to separate
% the .sty file from the LaTeX source template, so that people can
% more easily include the .sty file into an existing document.  Also
% changed to more closely follow the style guidelines as represented
% by the Word sample file. 

% Note that since 2010, USENIX does not require endnotes. If you want
% foot of page notes, don't include the endnotes package in the 
% usepackage command, below.

% This version uses the latex2e styles, not the very ancient 2.09 stuff.
\documentclass[letterpaper,twocolumn,10pt]{article}
\usepackage{usenix,epsfig,endnotes}

\usepackage[flushleft]{threeparttable}
\usepackage{amssymb}
\usepackage{graphicx}
\usepackage{amsmath}

\usepackage{url}

\newcommand{\keywords}[1]{\par\addvspace\baselineskip
\noindent\keywordname\enspace\ignorespaces#1}

\begin{document}

%don't want date printed
\date{}

% first the title is needed
\title{\Large \bf From Helios to Zeus}

\author{
{\rm Georgios Tsoukalas}\\
\and 
{\rm Kostas Papadimitriou}\\
\and 
{\rm Panos Louridas}\\
\and 
{\rm Panayiotis Tsanakas}\\
Greek Reseach and Education Network,\\
56 Mesogeion Avenue, Athens, Greece\\
\{gtsouk,kpap,louridas,tsanakas\}@grnet.gr
}

\maketitle

% Use the following at camera-ready time to suppress page numbers.
% Comment it out when you first submit the paper for review.
\thispagestyle{empty}

\subsection*{Abstract}

We present Zeus, an Internet ballot casting and counting system
based on Helios that allows elections in which ballots cannot be
tallied homomorphically. Zeus separates the production of election
results from voting and counting, thus allowing any type of
elections to be carried out, while maintaining the main features of
the original Helios system in terms of anonymity and verifiability.

\section{Introduction}

Helios is a well known system for Internet voting, in which the whole
process is carried out through digital means---there is no paper
trace, nor ballots in physical forms. It has been used in several real
world elections and its basic architectural design has proven robust.

In this paper we present Zeus, a system based on Helios, which extends
Helios's range of applications while reducing the actual work
performed by Helios and its role in the election process. Helios is a
system that performs \emph{ballot casting}, \emph{ballot counting},
and \emph{production of election results}. Zeus performs only the
first two, leaving the production of election results to other
systems, or to additional modules. In this way, it can accommodate
more voting systems than Helios, which currently supports only
approval-like voting.

\section{Roots and Rationale}

Zeus was initiated after the use of electronic vote was permitted by
decree for the election of the Governing Councils of Higher Education
Institutions in Greece. In each institution the Governing Council is
directly elected by its faculty and is its main governing body. The
election uses the Single Transferable Vote (STV) system, in which
voters do not simply indicate the candidates of their preference, but
also rank them in order of preference. 

When we were charged with providing an implementation of a system
implementing electronic voting we decided to investigate Helios's
suitability, as we needed a mature system with a proven record in real
world elections and published open source code. The current version of
Helios (version 3) allows internet election from end-to-end: from the
moment the voter casts a ballot through a web browser to the
publication of election results. It does that by never actually
decrypting the ballots but performing a series of homomorphic
calculations on them. In the end, the results of the calculations are
decrypted and published. 

Although using a single system for the whole process is appealing, the
use of homomorphic counting in Helios cannot accommodate voting
systems in which not just the individual choices on the ballot matter,
but the whole ballot itself. In STV, homorphical tallying could pass
to the STV algorithm the information that a certain candidate has been
selected in rank $r$ by $n$ voters, but this is not enough, as the whole
ballot and not just each rank separately is passed around during STV's
counting rounds.

We realised, however, that it is not necessary to use Helios's
homomorphic counting capabilities. We decided to use Helios for
\emph{counting the ballots}, not for producing the election results.
Once we do have a veriable ballots count, this can be fed to an STV
calculator, or indeed to a calculator of any voting system. Since the
ballots are published, and the algorithm is also published, a third
party can always verify that the results are correct.

Interestingly, the original publication of Helios~\cite{adida:2008}
did not use homomorphic tallying, but relied on mixnets for
guaranteeing voter anonymity. ~\cite{bulens:2011}

\section{Ballot Casting and Encoding}

In Helios, ballots consist of answers to binary ``yes'' or ``no''
questions, framed in the appropriate way. For instance, a voter
indicates $k$ out of $n$ candidates on a ballot by selecting them, in
which case the answer ``Would you like X as a Y?'' gets a yes (1),
otherwise a no (0). In STV, as in any system in which we would need
the whole ballot to be decrypted in the end, the ballot must be
encoded in some way. We encode each ballot as a integer by assigning a
unique number to each possible candidate selection and ranking. The
total number of possible ballots is $p _{n1} + p_{n2} + \cdots + p
_{nk}$, where $p_{nk}$ is the number of sequences of $k$ objects out
of $n$, that is $p_{nk} = n(n - 1)\cdots(n - k + 1)$\footnote{The
  number $p_{nk}$ is also written $n^{\underline{k}}$, or $(n)_k$,
  called ``Pochhammer's symbol''~\cite[p.\ 48]{graham:1994}. In closed
  form it is $p_{nk} = \sum_{k=0}^{n} (-1)^{n-k}\left\{n \atop
    k\right\}x^k$, where $\left\{n \atop k\right\}$ is the Stirling
  number of the first kind~\cite{weisstein:pochhammer}.}. As each
ballot is sent in encrypted form to the Zeus server, for its encoding
number $b$ we must have $b \in [0, 10^p]$, where $p$ is the order of
the group used in the ElGamal encryption scheme, so the encoding does
not present a practical limit in real elections (if it did, we could
always break the number in parts and send them separately).

The implementation of the encoding follows closely the mathematical
definition. We encode each ballot as an integer by enumerating the set
$\mathcal{E}$ of all possible ballots. We take $k$ choices out of $n$,
for $0\leq k \leq C$, where $C$ is the maximum choices allowed in the
ballots \footnote{$N=C$ for the elections we hosted}, and we then take
their $k!$ permutations. Summing it up, all the possible ballots are
\begin{equation}
\label{eq:max_encoded}
|\mathcal{E}| = \sum^{s}_{k=0}\binom{n}{k}k! \quad
\hbox{\small where} \quad 0\leq k \leq C
\end{equation}
When enumerating, we first count the smaller selections (i.e. take
\textit{zero} candidates, then \textit{one}, then \textit{two},
\ldots) so that for small numbers of choices the encoded ballot will
have a small value, thus saving valuable bit-space. \footnote{ For
  example, selecting up to all of 300 candidates needs 2048 bits,
  while selecting 10 out of 1000 candidates needs only 100 bits.}

% gtsouk: describe the exact encoding / decoding algorithm for the
% ballots. 

\section{The Cryptographic Model}

We have collected all cryptographic aspects of Zeus elections and
implemented them into a generic, standalone, and extensible software
module. \footnote{\texttt{zeus.core}} This module implements an
abstract version of the election workflow, all the required
cryptographic functions, and all validations of data and the workflow
itself. This module is fitted into the production environment by
software extensions (i.e. subclassing) where the web application uses
it as needed.

\subsection{Software Dependencies}
Random number generation \emph{(Fortuna)}, primality testing
\emph{(Miller-Rabin)} and modulo inversion from \texttt{PyCrypto}.
Exponentiation from \texttt{libgmp} via \texttt{gmpy} (really faster
than Python's builtin \texttt{pow}). We have implemented all other
operations in native Python and the Python standard library, including
hashing. We have studied \texttt{Helios}, \texttt{PloneVoteCryptoLib},
\texttt{PyCrypto} for our implementation.

% \subsection{Cryptosystem}
% Zeus uses the ElGamal cryptosystem on the prime-order-$q$ subgroup
% $\mathcal{G}$ of the quadratic residues of a safe prime $p = 2q + 1$,
% such that
% $$m \in \mathcal{G} \longleftrightarrow \mathcal{L}(m) = m^q = 1 \mod p$$
% $\mathcal{L}$ being the \emph{Legendre} symbol.
% For reference, we reproduce all the primitives we use in a table.
% All base numbers are in $\mathcal{G}$, all exponents in $\mathbb{Z}^{*}_q$,
% and all operations are $\mod p$, unless explicitly noted.
% We group $x=a, y=b, \ldots$ as $(x,y,\ldots)\equiv(a,b,\ldots)$.

% \begin{tabular}{rl}
% \textbf{modulus}       \, & $p \equiv 3(\mod 4)$ \hfill\textit{\small(safe prime)}\\
% \textbf{generator}     \, & $g: g^q = 1$      \\
% \textbf{order}         \, & $q = (q-1)/2$\hfill\textit{\small(ElGamal group prime order)}\\
% \textbf{secret}        \, & $x$               \\
% \textbf{public}        \, & $y = g^x$         \\
% \textbf{committment}   \, & $t$               \\
% \textbf{challenge}     \, & $c$               \\
% \textbf{response}      \, & $f$     \hfill\textit{\small(from proof)} \\
% \textbf{\parbox{7em}
%     {\setlength{\baselineskip}{.85\baselineskip}
%      \raggedleft
%      group \\
%      encoding}}        \, & $T: x \mapsto \left\{
%                             \begin{matrix}  x &,&\quad x^q=1 \\
%                                            -x &,&\quad x^q\neq 1
%                             \end{matrix}\right.\quad
%                             T^{-1}: e \mapsto \left\{
%                             \begin{matrix}  e &,&\quad e \leq q \\
%                                            -e &,&\quad e > q \\
%                             \end{matrix}\right.
%                             $ \\
% \textbf{secret nonce}  \, & $r, w$  \hfill\textit{\small(in encryptions and proofs)} \\
% \textbf{ciphertext}    \, & $(a, b)\equiv(g^r, y^r m$)
%                                     \hfill\textit{\small(encryption)} \\
% \textbf{plaintext}     \, & $m = a^{-x}b$
%                                     \hfill\textit{\small(decryption)} \\
% \textbf{reencryption}  \, & $(a',b')\equiv(g^{r'}a, y^{r'}b)$ \\
% \textbf{hashing}       \, & $\mathcal{H}(n_0n_1n_2\ldots)$
%                             \hfill\textit{\small(hash textual representation of numbers)} \\
% \textbf{discrete log}  \, & $y = g^x \Rightarrow \log_gy=x \;\leadsto\;$
%                                     \textbf{\small prove you know} $x$ \\
%                        \, & $(t, c, f) \equiv (g^w, \mathcal{H}(t), w+xc)$
%                                     \hfill\textit{\small(prove knowledge)} \\
%                        \, & $g^f \overset{?}{=} ty^c$
%                                     \hfill\textit{\small(verify knowledge)} \\
%                        \, & $g, y, u, v \;\leadsto\;$
%                                     \textbf{\small prove} $\:log_gu = log_yv = w$
%                                     i.e. $(g, y, g^w, y^w)$ \\
%                        \, & $(t_g,t_y,c,f) \equiv (g^w, y^w, \mathcal{H}(t_gt_y), w+xc)$
%                                     \hfill\textit{\small(prove equality)} \\
%                        \, & $g^f \overset{?}{=} t_gu^c \wedge
%                              y^f \overset{?}{=} t_yv^c$
%                                     \hfill\textit{\small(verify equality)} \\
% \textbf{signature}     \, & $(z, s)\equiv\big(g^{w}, w^{-1}(m-zx)\mod (p-1)\big)
%                                     \quad w = 2w'-1,\: 3\leq w'\leq q$ \\
%                        \, & $m^s \overset{?}{=} y^zz^s$
%                                     \hfill\textit{\small(verify signature)} \\
% % TODO: cite Fiat-Shamir, Schnorr, Chaum-Pedersen, HAC in this table
% \end{tabular}

% \subsection{Creating}
% \subsection{Voting}
% \subsection{Mixing}
% \subsection{Decrypting}
% \subsection{Validation}

\section{Election User Interface}

The users of Zeus were not expected to be experts in cryptography, or
to have any knowledge of computer science; we could only assume
familiarity with internet browsing, since the electoral body comprises
people with heterogeneous characteristics. A significant design goal
was therefore \emph{interface}, and \emph{workflow} simplicity. At the
same time, the knowledgeable voter needed access to all information
and functions needed to both understand and verify the process.

\subsection{Zeus Users}

Zeus distinguishes the following classes of users:

\begin{itemize}
\item Election administrators
\item Election committee members
\item Voters
\end{itemize}

Election administrators are responsible for setting up an election;
that is, starting a new election instance in Zeus, setting the date
for the election, entering the particulars of the election committee
members and the list of voters. During the election the administrators
are responsible for updaging the list of voters (i.e., adding,
removing, or updating the details for a voter) and extending the poll
times, if necessary. When polls close the administrators are
responsible for starting the ballot mixing and vote count process.
Election committee members hold the cryptographic keys for the
election. Before the election starts they have to create their keys
and upload the public parts to the Zeus server. During decryption each
election committee member partially decrypts the ballots using the
private part of the key. Voters visit the voting booth to cast their
vote; they can vote repeatedly until polls close.

Election administrators access Zeus through a username and password.
Election committee members and voters access Zeus through URL links
that are sent to them by the system, as we see below.

% \subsection{Administrator View}
% \subsection{Voting Booth}
% \subsection{Auditing}

% \section{Ballot Submission}

% The submitted ballot contains
% %gtsouk: describe the conntents of the submitted ballot
% A submitted ballot is a JSON object of the form:
% \begin{verbatim}
% {
%   answers : [];
%   election_hash : ;
%   election_uuid : ;
% }
% \end{verbatim}

% \section{Election Verification}

% %gtsouk: describe the election verification algorithm

\section{The Election Workflow in Zeus}

The election workflow in Zeus consists of the following phases:
election preparation, polling, and tallying.

\subsection{Preparing the Election}

\subsubsection{Initialisation}

The election administrator initialises the election in Zeus by
entering the polling date (or dates), the members of the election
committee, the name of the election and any other information required
for defining the process. The election administrator may be a member
of the election committee.

\subsubsection{Notifications to the Election Committee}

The election administrator uses Zeus to send notifications to the
election committee members. The notifications contain URLs by which
the committee members can access the system in order to generate and
upload their encryption keys.

\subsubsection{Generation of Election Keys}

Committee members generate their encryption keys. The encryption keys
are generated in their browsers and the members are instructed to save
them in a secure place. They then upload the public parts of their
keys to Zeus, to be used for encryption the voters' ballots when polls
open. 

\subsubsection{Ballot Setup}

The election administrator in coordination with the election committee
sets up the election ballots. This includes defining the number of
ballot questions and acceptable answers, whether the order of ballot
choices will be important (as in STV) or not (as in approval voting),
and the contents of each ballot.

\subsubsection{Voters Addition}

Voters are entered to the system by the administrator and the election
committee by means of a CSV file (typically generated through a
spreadsheet application) containing, for each voter the voter's email
and name.

\subsubsection{Election Freeze}

Finally the election is frozen; when this happens, the voters receive
at their mailboxes a message inviting them to cast their vote. The
message contains a link that leads them to the voting booth, where
they can compile their ballot. The voting booth opens at the date and
time appointed by the election authority; if the voters visit the link
before, they are informed accordingly.

\subsection{Voting}

Voters access the voting booth via the link they have received. They
compose their ballot, which they proceed to submit. All the process
happens at the client browser, with no interaction with the server
until the ballot has been encrypted and sent to be stored along with
the rest of the election's ballots. The user then receives an e-mail
containing a receipt for the ballot. The receipt contains the
cryptographic data that can be used to verify that the ballot has been
counted in the results.

\subsection{Tallying}

When the polls close the administrator and the election committee can
proceed to mix and decrypt the ballots. Mixing is carried out at least
once, on the Zeus server, and can optionally be carried out in
additional independent servers. Independent mixing works by means of a
command line tool to which a URL with the encrypted ballots is passed;
the tool mixes them and returns them back to Zeus. 

Once all mixes have been completed, the election committee members
receive an e-mail notification to proceed to decrypt the ballots. The
notification contains a URL link through which the encrypted ballots
are downloaded to the client browser, partially decrypted there, and
returned back to Zeus.

After the decryption Zeus provides a tally of the decrypted and
anonymised ballots, along with all the cryptographic proofs that can
be used to verify and audit and election.

% \section{Implementation Considerations}

\section{Experience}

From the voter's point of view, there has been a published usability
analysis of Helios~\cite{karayumak:2011}. From our part we did not
receive any particular complains from the voters, except for one: that
the system would not work with Internet Explorer. The users were
informed of that in the notifications they received, but apparently
not all of them noticed it, or knew exactly how to download and
install one of the supported browsers (recent editions of Firefox,
Chrome).

Zeus was used in 23 elections that took place in 22 institutions
around Greece (see Table~\ref{table:zeus-elections}). Elections were
carried out successfully in all planned institutions. This was a
significant success, as the stakes were particularly high and the
voting process charged with emotions. To understand that it is
necessary that we provide some surrounding context.

The law that instituted the Governing Councils in Greek educational
institutions met resistance from some political parties, student and
academics unions. A widespread means of protest was to disrupt the
election of Governing Councils, so that they could not be inaugurated.
Successive aborted elections pushed forward the adoption of electronic
voting, which in turn was the subject of controversy. This led to
attacks on Zeus, and the voting process supported by Zeus. None of
them was eventually successful, and none of them targeted Zeus per se.
We describe them briefly in what follows.

\subsection{University of Thrace}

The elections at the University of Thrace had been planned for October
24, 2012. Due to a coding bug, the polls opened the at the time the
election was frozen, on October 22. This was discovered by some
voters, who went on to vote before the official poll opening. The
issue was publishised and the election was annulled and repeated at a
later date. Technically, the election could have proceeded without a
problem, since there was no way anybody could have tampered with the
election results, but the issue was sensitive and politicised, so that
the election committee took the most cautious route.

Elections were called again for October 29, 2012, with polls opening
at 9:00. At dawn our Networks Operations Center noticed a large number
of connections to Zeus. Upon investigation, it was found that Zeus was
the subject of a slowloris attack~\cite{slowloris}. The attach was
swiftly dealt with, and apart from some initial inconvenience did not
cause any real problems. Moreover, the attackers used static IP
addresses and could easily be traced and blocked.

\subsection{University of the Aegean}

The University of the Aegean had planned its elections for November 2,
2012. After freezing the election, and when the notifications to the
voters had been sent, voters started receiving additional
notifications containing bogus URLs, from a cracked university server.
Although the bogus URLs were not functional, a number of users were
confused. The electoral committee decided to cancel the elections and
call new elections at a later date. To avoid similar situations in
coming elections we recommended the use of Sender Policy Framework
(SPF)~\cite{rfc4408} to institutions that had not been using it.

The repeat elections were held on November 9, 2012. The authorities at
the university had authorised the set up of a new mail server, to be
used solely for the elections, in which all voter's were opened
accounts. The mail server was hosted outside the university's
premises. This avoided a problem that would have risen from a sit-in
at the building housing the main mail server, which had been switched
off by protesters at election day.

\subsection{Agricultural University of Athens}

On election day, November 5, 2012, protesters staged a sit-in at the
mail server building of the Agricultural University of Athens. The
mail server was switched off, cutting off access to e-mail, and
depriving voters that had not downloaded the access link from voting.
The university authorities responded by asking voters to come
physically to a location outside the campus, where they could be
issued with a new URL.

\subsection{University of Patras}

The University of Patras held its elections on November 7, 2012. Just
prior to the opening of the polls a slowloris attack as noticed and
dealt with.

\subsection{University of Athens}

A little while after polls opened at the University of Athens, on
November 12, 2012, protesters staged a sit-in at the university's
Network Operations Server and cut-off internet access. This took
offline mail access to members of the university and, like in the
Agricultural University of Athens, deprived voters that had not
downloaded the access link from voting. More dramatically, the
internet shutdown affected all sorts of university services, including
internet connectivity to university hospitals.

The university responded by extending polling by three days. At the
same time, an alternative voter's notification scheme was implemented,
via which the voters would receive instructions for voting via SMS.
The sit-in was lifted on the last election day, as it was realised
that the elections would go on regardless.

\subsection{National Technical University of Athens}

A sit-in similar to the one at the University of Athens was planned at
the National Technical University of Athens. However, it did not last
long, as the SMS-based infrastructure was already in place, so that it
would be irrelevant, and the university authorities made it clear that
they would not tolerate the disruption.

A different, more insidious kind of attack was revealed later on. A
few days before the election, one voter had contacted us complaining
that the link he had received did not seem to be functioning at all
(it should link to a page notifying the user about the coming
elections; instead, the user was redirected to the main Zeus page).
Two days before the election it was discovered that the user's e-mail
stored in Zeus had been changed, and a new e-mail had been sent to the
newly entered mail address. The user complained that he never used
that adddress, so another URL was issued and sent to his institutional
mail address. The voter voted without a problem on election day.

A few days afterwards a protesters' site announced they had
circumvented Zeus security by taking hold of a voter's URL after
contacting the election committee and asking them to update the
voter's contact details. That explained the e-mail change, and it
meant that it was the result of a social engineering attack. Luckily,
it had been caught in time---something the protesters did not let on.

\begin{table*}[t]
  \begin{threeparttable}
    \begin{tabular}{llllll}
      \hline
      Institution & Voters & Voted & Ballots & Open & Close\\
      \hline
      Harokopio University & 61 & 59 & 63 & 2012-10-19 10:00 & 2012-10-19 11:00\\
      TEI Piraeus & 144 & 102 & 111 & 2012-10-22 09:00 & 2012-10-22 13:00\\
      Ionian University & 105 & 92 & 101 & 2012-10-25 09:00 & 2012-10-25 15:00\\
      TEI Patras & 93 & 78 & 79 & 2012-10-26 09:00 & 2012-10-26 13:00\\
      University of Thrace & 583 & 490 & 528 & 2012-10-29 09:00 & 
      2012-10-29 16:00\\
      TEI Athens & 487 & 437 & 470 & 2012-10-31 08:00 & 2012-10-31 18:00\\
      University of Thessaly & 429 & 328 & 345 & 2012-10-31 09:00 & 
      2012-10-31 14:36\\
      Panteion University & 239 & 181 & 183 & 2012-10-31 09:00 & 
      2012-10-31 16:00\\
      University of Thessaloniki & 2065 & 1580 & 1632 & 2012-11-01 09:00 
      & 2012-11-01 16:00\\
      Athens University of Economics and Business & 195 & 172 & 181 & 
      2012-11-02 10:00 & 2012-11-02 17:00\\
      Agricultural University of Athens & 182 & 102 & 110 
      & 2012-11-05 09:00 & 2012-11-05 16:00\\
      University of Ioanina & 536 & 426 & 443 & 2012-11-05 09:00 &
      2012-11-05 17:00\\
      University of Crete & 493 & 361 & 373 & 2012-11-05 09:00 & 
      2012-11-05 18:00\\
      University of Macedonia & 169 & 163 & 172 & 2012-11-06 10:00 & 
      2012-11-06 18:00\\
      University of Patras & 703 & 532 & 555 & 2012-11-07 09:00 & 
      2012-11-07 17:00\\
      Athens School of Fine Arts & 46 & 40 & 41 & 2012-11-08 07:00 & 
      2012-11-08 16:00\\
      University of the Aegean & 292 & 161 & 172 & 2012-11-09 09:00 & 
      2012-11-09 17:00\\
      University of Athens & 1897 & 1504 & 1618 & 2012-11-12 09:00 & 
      2012-11-15 18:00\\
      University of Piraeus & 183 & 180 & 199 & 2012-11-14 09:00 & 
      2012-11-14 15:00\\
      University of the Peloponnese & 128 & 124 & 128 &
      2012-11-22 07:00 & 2012-11-22 17:00\\
      Technical University of Crete & 126 & 116 & 127 & 
      2012-12-04 10:00 & 2012-12-04 15:00\\
      National Technical University of Athens & 536 & 351 & 363 &
      2012-12-10 09:00 & 2012-12-10 17:00\\
      TEI Piraeus & 143 & 104 & 105 & 2012-12-21 09:00 & 2012-12-21 17:00\\
    \end{tabular}
    \begin{tablenotes}
      \small 
      \item The number of ballots may be larger than the number of
      people voted when voters have voted multiple times.
      \item TEI is Technological Educational Institute. 
    \end{tablenotes}
  \end{threeparttable}
  \caption{Zeus Held Elections\label{table:zeus-elections}}
\end{table*}

% gtsouk: Including timing measurements, especially wrt differen versions of
% Python, GMP, etc.

% gtsouk: also instances where existing implementations were not up to
% par from a cryptographic point of view

\section{Availability}

Zeus is open software, available at \url{https://github.com/grnet/zeus}.

{\footnotesize
\bibliographystyle{acm}
\bibliography{zeus}
}


\end{document}


