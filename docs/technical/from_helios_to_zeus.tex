\documentclass[runningheads,a4paper]{llncs}

\usepackage{amssymb}
\setcounter{tocdepth}{3}
\usepackage{graphicx}

\usepackage{url}
\urldef{\mails}\path|{gtsouk, kpap, louridas, tsanakas}@grnet.gr|

\newcommand{\keywords}[1]{\par\addvspace\baselineskip
\noindent\keywordname\enspace\ignorespaces#1}

\begin{document}

\mainmatter  % start of an individual contribution

% first the title is needed
\title{From Helios to Zeus}

\titlerunning{From Helios to Zeus}

\author{Georgios Tsoukalas%
\and Kostas Papadimitriou%
\and Panos Louridas%
\and Panayiotis Tsanakas}

\authorrunning{Tsoukalas, Papadimitriou, Louridas and Tsanakas}

\institute{Greek Reseach and Education Network,\\
56 Mesogeion Avenue, Athens, Greece\\
\mails\\
\url{http://www.grnet.gr}}

\maketitle

\begin{abstract}
  We present Zeus, an Internet ballot casting and counting system
  based on Helios that allows elections in which ballots cannot be
  tallied homomorphically. Zeus separates the production of election
  results from voting and counting, thus allowing any type of
  elections to be carried out, while maintaining the main features of
  the original Helios system in terms of anonymity and verifiability.
  \keywords{Internet voting, voting systems, electronic ballots}
\end{abstract}


\section{Introduction}

Helios is a well known system for Internet voting, in which the whole
process is carried out through digital means---there is no paper
trace, nor ballots in physical forms. It has been used in several real
world elections and its basic architectural design has proven robust.

In this paper we present Zeus, a system based on Helios, which extends
Helios's range of applications while reducing the actual work
performed by Helios and its role in the election process. Helios is a
system that performs \emph{ballot casting}, \emph{ballot counting},
and \emph{production of election results}. Zeus performs only the
first two, leaving the production of election results to other
systems. In this way, it can accommodate more voting systems than
Helios, which currently supports only approval-like voting.

\section{Roots and Rationale}

Zeus was initiated after the use of electronic vote was permitted by
decree for the election of the Governing Councils of Higher Education
Institutions in Greece. In each institution the Governing Council is
directly elected by its faculty and is its main governing body. The
election uses the Single Transferable Vote (STV) system, in which
voters do not simply indicate the candidates of their preference, but
also rank them in order of preference. 

When we were charged with providing an implementation of a system
implementing electronic voting we decided to investigate Helios's
suitability, as we needed a mature system with a proven record in real
world elections and published open source code. The current version of
Helios (version 3) allows internet election from end-to-end: from the
moment the voter casts a ballot through a web browser to the
publication of election results. It does that by never actually
decrypting the ballots but performing a series of homomorphic
calculations on them. In the end, the results of the calculations are
decrypted and published. 

Although using a single system for the whole process is appealing, the
use of homomorphic counting in Helios cannot accommodate voting
systems in which not just the individual choices on the ballot matter,
but the whole ballot itself. In STV, homorphical tallying could pass
to the STV algorithm the information that a certain candidate has been
selected in rank $r$ by $n$ voters, but this is not enough, as the whole
ballot and not just each rank separately is passed around during STV's
counting rounds.

We realised, however, that it is not necessary to use Helios's
homomorphic counting capabilities. We decided to use Helios for
\emph{counting the ballots}, not for producing the election results.
Once we do have a veriable ballots count, this can be fed to an STV
calculator, or indeed to a calculator of any voting system. Since the
ballots are published, and the algorithm is also published, a third
party can always verify that the results are correct.

Interestingly, the original publication of Helios~\cite{adida:2008}
did not use homomorphic tallying, but relied on mixnets for
guaranteeing voter anonymity. 

\section{Ballot Casting and Encoding}

In Helios, ballots consist of answers to binary ``yes'' or ``no''
questions, framed in the appropriate way. For instance, a voter
indicates $k$ out of $n$ candidates on a ballot by selecting them, in
which case the answer ``Would you like X as a Y?'' gets a yes (1),
otherwise a no (0). In STV, as in any system in which we would need
the whole ballot to be decrypted in the end, the ballot must be
encoded in some way. We encode each ballot as a integer by assigning a
unique number to each possible candidate selection and ranking. The
total number of possible ballots is $p _{n1} + p_{n2} + \cdots + p
_{nk}$, where $p_{nk}$ is the number of sequences of $k$ objects out
of $n$, that is $p_{nk} = n(n - 1)\cdots(n - k + 1)$\footnote{The
  number $p_{nk}$ is also written $n^{\underline{k}}$, or $(n)_k$,
  called ``Pochhammer's symbol''~\cite[p.\ 48]{graham:1994}. In closed
  form it is $p_{nk} = \sum_{k=0}^{n} (-1)^{n-k}\left\{n \atop
    k\right\}x^k$, where $\left\{n \atop k\right\}$ is the Stirling
  number of the first kind~\cite{weisstein:pochhammer}.}. As each
ballot is sent in encrypted form to the Zeus server, for its encoding
number $b$ we must have $b \in [0, 10^p]$, where $p$ is the order of
the group used in the ElGamal encryption scheme, so the encoding does
not present a practical limit in real elections (it it did, we could
always break the number in parts and send them separately).

\section{Election Verification}

\section{Implementation Considerations}

\bibliographystyle{splncs}
\bibliography{zeus}

\end{document}


